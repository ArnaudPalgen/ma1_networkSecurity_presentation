\documentclass[compress]{beamer}
\usepackage{MnSymbol,wasysym}
\usepackage[utf8]{inputenc}
\usepackage{amssymb}
\usepackage{eurosym}
\usetheme[navigation]{UMONS}
\setbeamertemplate{navigation symbols}{}
%\usetheme[navigation, no-subsection, no-totalframenumber]{UMONS}

\newcommand{\IR}{\mathbb{R}}


\title{Network security and management}
\subtitle{Digital signature}
\author{Dieudonné KASONGO KABASELE, Nikola JOVICIC, Arnaud PALGEN, Yacine SAHLI}
\institute[University of Mons]{%
  Electromagnetism and telecommunications department\\
  University of Mons
  \\[2ex]
  %\includegraphics[scale=0.5]{image}
  \includegraphics[height=4ex]{UMONS}\hspace{2em}%
  \includegraphics[height=4ex]{UMONS_FPMS}\hspace{2em}
  \raisebox{-1ex}{\includegraphics[height=6ex]{UMONS_FS}}\hspace{2em}
  
}
\begin{document}

\begin{frame}[plain]
  \titlepage
\end{frame}

\begin{frame}
  \tableofcontents
\end{frame}

\section{Introduction}
    \subsection{Our approach to the problem ...}
    \subsection{What is a Signature ?}
    \subsection{Digital Signature Function}
    \subsection{Difference between electronic and digital signature}

\begin{frame}{Our approach to the problem ...}
    \begin{enumerate}
        \item Understand what is fundamentally a signature and what it implies (conditions to respect)
        \item Common points between electronic and digital signature
        \item Explain the basic mechanisms behind digital signature
        \item Find existing solutions and compare them to the basic principle
        \item Compare solutions between them (pros and cons)
    \end{enumerate}
\end{frame}

\begin{frame}{What is a Signature ?}
There are two kinds :
    \begin{itemize}
    \item Electronic signature or e-signature\\
refers to data in electronic form, 	which is logically associated with other data in electronic form and which 	is used by the signatory to sign or confirm their approval of a document 	or transaction. 
 \item Digital signature\\
 These are a subset of electronic signatures because they are also in 	electronic form. They are a cryptographic mechanism often used to implement electronic signatures. Digital signatures go much 	further in terms of providing security and trust services.
    \end{itemize}
    \tiny{Source:\url{https://www.4point.com/blog/2017/06/what_is_an_e-signatu.html}}
\end{frame}

\begin{frame}{Digital Signature Function}
    \begin{itemize}
    \item Properties\\

  -Identification\\ 
  -Integrity\\ 
  
  \item Conditions\\

  -Authentic: the identity of the signatory must be traceable with certainty;

  -Forgery-proof: the signature cannot be forged. Someone cannot pretend to be someone else;

  -Not reusable: It is part of the signed document and cannot be moved to another document;

  -Unalterable: Once it is signed, it cannot be modified;

  -Irrevocable: the person who signed cannot deny it.
 
    \end{itemize}
\tiny{Source: \url{https://fr.wikipedia.org/wiki/Signature_num\%C3\%A9rique}}  
\end{frame}

\begin{frame}{Difference between electronic and digital signature}
   % \item Difference between electronic and digital signature
   \begin{center}
        \includegraphics[scale=0.5]{Signature}
   \end{center}
\tiny{Source: \url{https://acrobat.adobe.com/content/dam/doc-cloud/fr/pdfs/adobe-sign-electronic-and-digital-signatures-wp-fr.pdf}}
\end{frame}

    \subsection{}

\section{Basic principles of digital signatures}
\subsection{Diagram of the basic principle}
\begin{frame}
  \frametitle{Diagram of the basic principle}
    \begin{center}
    \includegraphics[scale=0.32]{digital_signature_principle_english}
    \end{center}
\tiny{Source: \url{https://fr.wikipedia.org/wiki/Signature_num\%C3\%A9rique}}
\end{frame}

\subsection{The hash function}
\begin{frame}
  \frametitle{The hash function}
  \begin{itemize}
  \item A hash function H takes a long string M as an input and outputs a different and shorter string \textit{h}, called the hash
    \begin{itemize}
    \item \textit{h} = H(M)
    \item Example with SHA1 : "\textit{Hello friend ! It is a beautiful day today, isn't it ?}" outputs "\textit{73341cf4db5967d8ac42d5061d4f4b4c6ab666e1}"
    \item The hash is usually expressed in an hexadecimal form
    \end{itemize}
  \item Examples of a hash functions:
    \begin{itemize}
    \item MD5, SHA1, SHA256, SHA384, ...
    \item $\textit{h} = H_{\oplus}(M) = M_1 \oplus M_2 \oplus M_3\oplus ... \oplus M_N$
    \end{itemize}  
  \item \textbf{Key features:}
        \begin{itemize}
    \item  computing \textit{h} = H(M) is easy but finding M from the hash should be very hard (in a reasonable time) !
    \item  a single change in the original in the original string shall output a completely different hash (e.g. : "Hello friend\textbf{\underline{s}} ! ..." outputs "\textit{aff42bd158ec3c095ca49f6d11c6f9961eb1de74}")
    \end{itemize} 
   
  \end{itemize}
\end{frame}

\subsection{Public certificates}
\begin{frame}
  \frametitle{Public certificates}
  \begin{itemize}
  \item Delivered by a \textbf{Certificate Authority}: trusted entity that issues and revokes public-key certificates and certifies the ownership of a public key by the "subject" of the certificate
  \item X.509 standard specified the format of these certificates:
  \begin{itemize}
      \item Version: version of the certificate
      \item Serial number: unique number linked to the certificate
      \item Algorithm: hash function and public-key encryption algorithm
      \item Issuer: the entity that issued the certificate
      \item Validity period
      \item Subject: owner of the certificate
      \item Public key
      \item Properties: Encrypted hash value of the certificate with the owner's private key
      \item Extension: additional information (used from version 3 and onward)
  \end{itemize}
  \item Comodo, Sectigo, Symantec, Geotrust, RapidSSL, ...
  \end{itemize}
\end{frame}




\section{Existing solutions}
\subsection{Adobe Sign}
\begin{frame}{What's Adobe Sign ?}
        Adobe Sign is the Adobe's proprietary solution for signing documents.\\
        Adobe Sign is compatible with
        \begin{itemize}
            \item Adobe PDF
            \item Microsoft Word, Excel, PowerPoint
            \item Text, Rich Text
            \item Images
            \item Web
        \end{itemize}
        Adobe Sign supports different types of authentication: Adobe Sign ID and Adobe ID, Google ID, Single Sign-On (SSO).
        
        The prices are:
        \begin{itemize}
            \item 145,05\euro/year for a person
            \item minimum 508,05\euro/year for a small company
        \end{itemize}
        
\end{frame}
    
    \begin{frame}{Security of Adobe Sign}
        \begin{enumerate}
            \item Adobe is certified ISO 27001, SOC2, PCI DSS.
            \item Adobe Sign uses a PKI to certify documents with numeric signature before distributing them to participants.
            \item Adobe use private virtual cloud of Amazon Web Services and Microsoft Azure.
            \item Adobe use encryption algorithms that comply with the PCI DSS standard. They encrypt the datas with AES 256 bits.
            \item They use HTTPS TLS V1.2 to transmit datas.
        \end{enumerate}
        
    \end{frame}
    \subsection{GnuPg}
        \begin{frame}{GnuPG}
        features:\\
        \begin{itemize}
            \item Hybrid ciphers
            \item encryption
            \item signature
            \item key management
        \end{itemize}
\end{frame}

        \begin{frame}{GnuPG encryption}
    \begin{block}{Key generation an listing}
        gpg -\--gen-key\\
        gpg -\--list-keys
    \end{block}
    \pause
    \begin{block}{Export /Import public key}
        gpg -\--export -\--armor Bob \textgreater publickey.asc\\
        gpg -\--import publickey.asc
    \end{block}
    \pause
    \begin{block}{Encrypt using public key}
        gpg -\--encrypt -\--recipient receiversname filename.txt\\
        gpg -\--encrypt -r raman -r steve -r gopi  a.txt
    \end{block}
    \pause
    \begin{block}{Decrypt using private key}
        gpg filename.txt.gpg
    \end{block}
\end{frame}

\begin{frame}{GnuPG signature}
    \begin{block}{Sign and verify signature}
        gpg -\--sign file.txt\\
        gpg -\--verify file.txt.gpg
    \end{block}
    \pause
    \begin{block}{Extract document}
        gpg -\--output doc.txt -\--decrypt file.txt.gpg
    \end{block}
    \pause
    \begin{block}{Clear sign}
        gpg -\--output file.sig -\--clearsign file.txt
    \end{block}
    \pause
    \begin{block}{Sign and encrypt }
        gpg -\--sign -\--encrypt --recipient recipient file.txt
    \end{block}
\end{frame}
    
\subsection{Microsoft Office: Word}
\begin{frame}
  \frametitle{Microsoft Office: Word}
  \begin{itemize}
    \item Provides trustworthy digital signature of documents, using digital certificates from entrusted CAs, such as GlobalSign or IdenTrust
    \item Easy to setup in the document (Insert $\rightarrow$ Signature Line)
    \item Requires to pay for certificates (10-30 euros per month - several hundreds of euros per year !!)
    \item Practical example: \url{https://piv.idmanagement.gov/userguides/signworddoc/}
    \item GlobalSign yearly subscriptions: \url{https://www.globalsign.com/en/microsoft-office-document-signing/}
    \item Mostly oriented for companies who can afford the fees of the digital certificates
  \end{itemize}
\end{frame}
    
    \subsection{Online solutions}
        \begin{frame}{Online solutions}
        There are many free or paid online solutions for signing documents.\\
        They are easy to use but we have no information on the security of these signatures. We can see an example \href{https://smallpdf.com/fr/signer-un-pdf}{\beamergotobutton{here}}
        \end{frame}

    \subsection{LaTeX}
        \begin{frame}{LaTex}
        The package \textit{eforms} provide commands to add signature into a pdf.
        We can use the command\\
        \textit{\\sigField[1]\{2\}\{3\}\{4\}} to add a signature.\\
            Parameter Description:
            \begin{enumerate}
                \item optional, used to enter any modification of appearance/actions
                \item the title of the signature field
                \item the width of the bounding rectangle
                \item the height of the bounding rectangle
            \end{enumerate}
            
            The package uses Acrobat to sign the document.

        \end{frame}
    

\section{Conclusions}

\begin{frame}{Conclusion}
    \begin{itemize}
        \item Adobe Sign
            \begin{itemize}
                \item reliable, safe and easy to use \smiley{}
                \item expensive (not available in the free version) \frownie{}
            \end{itemize}
        \item GnuPg
            \begin{itemize}
                \item free, secure and available for Windows Linux and MacOS \smiley{}
                \item can become complex \frownie{}
            \end{itemize}
        \item Microsoft Word
            \begin{itemize}
                \item simple to set up and reliable \smiley{}
                \item certificates are expensive \frownie{}
            \end{itemize}
        \item Online solutions
            \begin{itemize}
                \item free and easy to use \smiley{}
                \item not secure  \frownie{}
            \end{itemize}
        \item LaTeX
            \begin{itemize}
                \item free and relatively easy to set for the signer \smiley{}
                \item not as secure as the previous ones \frownie{}
            \end{itemize}
    \end{itemize}
\end{frame}

\begin{frame}
    \begin{center}
        \vfill
        {\huge Thank you for your attention !}
        \vfill
    \end{center}
\end{frame}

\end{document}